%! LaTeX Vorlage
\documentclass[12pt, fleqn, captions=nooneline, titlepage, footsepline, headsepline, toc=sectionentrywithdots, listof=entryprefix, bibliography=totoc, parskip=half-]{scrartcl}

\usepackage{silence} %unnötige Warnungen unterdrücken
\WarningFilter{latex}{You have requested}
\WarningFilter{scrlayer-scrpage}{\headheight to low}
\WarningFilter{scrlayer-scrpage}{\footheight to low}
\WarningFilter{scrlayer-scrpage}{Very small head height detected}
\WarningFilter{fvextra}{}
\WarningFilter{lineno}{}


\usepackage{tocbasic}
\usepackage[ngerman]{babel}
%\usepackage[backend=biber, style=authortitle, isbn=false]{biblatex}

%! Das Inhaltsverzeichnis wird an dieser Stelle formatiert.
\RedeclareSectionCommands[beforeskip=-.1\baselineskip, afterskip=.1\baselineskip, tocindent=0pt, tocnumwidth=45pt]{section,subsection,subsubsection}

%! Formatierung aller Verzeichnisse
\renewcaptionname{ngerman}{\refname}{Quellenverzeichnis}
%\setuptoc{toc}{totoc}
\setuptoc{lof}{totoc}
\setuptoc{lot}{totoc}
\renewcommand*\listoflofentryname{\bfseries\figurename}
\BeforeStartingTOC[lof]{\renewcommand*\autodot{\space\space\space\space}}
\addtokomafont{captionlabel}{\bfseries}
\renewcommand*\listoflotentryname{\bfseries\tablename}
\BeforeStartingTOC[lot]{\renewcommand*\autodot{\space\space\space\space}}

%? Anhangsverzeichnis
\providecaptionname{ngerman}{\listofatocentryname}{Anhang}
\DeclareTOCStyleEntry[level=0, indent=50pt, numwidth=5cm, entrynumberformat={Anlage}]{tocline}{atoc}

\makeatletter
\AfterTOCHead[atoc]{\let\if@dynlist\if@tocleft}
\newcommand*{\useappendixtocs}{
  \renewcommand*{\ext@toc}{atoc}
  \RedeclareSectionCommands[tocindent=0pt]{section, subsection, subsubsection}
  \RedeclareSectionCommands[tocnumwidth=85pt]{section, subsection, subsubsection}
  \addtokomafont{sectionentry}{\mdseries}
  \renewcommand*\listoflofentryname{\mdseries}
  \renewcommand{\thesection}{\arabic{section}}
  \renewcommand{\@dotsep}{10000}
  }
\newcommand*{\usestandardtocs}{
  \renewcommand*{\ext@toc}{toc}
  }
\makeatother

%! Ermöglicht die Ausgabe des aktuellen Titels
\usepackage{nameref}
\makeatletter
\newcommand*{\currentname}{\@currentlabelname}
\makeatother
\usepackage{xcolor}


\usepackage[normalem]{ulem}
\useunder{\uline}{\ul}{}
%! Latex-Packages, die verwendet werden
\usepackage{amsmath}
\usepackage{amssymb}
\usepackage{amsthm}
\usepackage{tabularx}
\usepackage{multirow}
\usepackage{setspace} 
\usepackage{booktabs}
\usepackage{svg}
\usepackage{graphicx}
\usepackage{float}
\usepackage[a4paper,lmargin={2.5cm},rmargin={2.5cm},tmargin={2cm},bmargin={2cm}]{geometry}
\usepackage{lineno}
\usepackage{csquotes}

%! Code Integration im Dokument
%? Inklusive Erzeugung eines Custom Enviroments für Programmcodes
\usepackage[outputdir=latex.out]{minted}
%\SetupFloatingEnvironment{listing}{name=Program code}
%\SetupFloatingEnvironment{listing}{listname=List of Program Code}


\DeclareNewTOC[
  type=code,                         % Name der Umgebung
  types=codes,                       % Erweiterung (\listofschemes)
  float,                               % soll gleiten
  floatpos=H,
  tocentryentrynumberformat=\bfseries,                        % voreingestellte Gleitparameter
  name=Code,                         % Name in Überschriften
  listname={Programmcodeverzeichnis}, % Listenname
  % counterwithin=chapter
]{loc}
\setuptoc{loc}{totoc}

\renewcommand*{\codeformat}{%
  \codename~\thecode%
  \autodot{\space\space\space}
}
\BeforeStartingTOC[loc]{\renewcommand*\autodot{\space\space\space\space}}


%! Lang ersehntes Formelumgebung / Formelverzeichnis
\DeclareNewTOC[
  type=formel,                         % Name der Umgebung
  types=formeln,                       % Erweiterung (\listofschemes)
  float,                               % soll gleiten
  tocentryentrynumberformat=\bfseries,                        % voreingestellte Gleitparameter
  name=Formel,                         % Name in Überschriften
  listname={Formelverzeichnis}, % Listenname
  % counterwithin=chapter
]{lom}
\setuptoc{lom}{totoc}
\renewcommand*{\formelformat}{%
  \formelname~\theformel%
  \autodot{\space\space\space}
}
\BeforeStartingTOC[lom]{\renewcommand*\autodot{\space\space\space\space}}


%! Schriftart
\usepackage{helvet}
\usepackage{microtype}
\renewcommand{\familydefault}{\sfdefault}
%! Hyperref und PDF-meta
\usepackage[hidelinks]{hyperref}
\usepackage[numbered]{bookmark}
\usepackage{acronym}
\usepackage{enumitem}

%Kein Umbruch auf der titelseite
\let\endtitlepage\relax

\renewcommand*{\aclabelfont}[1]{\acsfont{#1}} %Abkürzungsverzeichnis - Formatierung
 %zB zum korrekten anzeigen von BASH-Befehlen


%! Caption um Tabellen und Abbildungen richtig zu beschriften
\usepackage{caption}
% Captions linksbündig auch wenn einzeilig
\captionsetup{
  labelsep=none,
  justification=raggedright,
  singlelinecheck=false
}
\renewcommand*{\figureformat}{%
  \figurename~\thefigure%
  \autodot{\space\space\space}
}
\renewcommand*{\tableformat}{%
  \tablename~\thetable%
  \autodot{\space\space\space}
}


%! Formatierung der Fußnotenzitate / Literaturverzeichnis


%! Kopf- und Fußzeile
\usepackage[automark]{scrlayer-scrpage} 
\pagestyle{scrheadings} 
\clearmainofpairofpagestyles 
\clearplainofpairofpagestyles
\renewcommand*{\sectionmarkformat}{}
\rohead{\textnormal{\headmark}}
%\lohead{\includegraphics[height=8mm]{bilder/logo.png}}
\lohead{}
\lofoot{}
\cofoot{} 
\rofoot{\textnormal{Seite}~\pagemark}
\makeatletter
\usepackage{geometry}
\setlength{\footheight}{18.85002pt}
\geometry{a4paper,
          left=25mm,right=25mm,top=20mm,bottom=20mm,
          includehead=false,
          includefoot=false,
          headheight = \baselineskip,
          headsep = \dimexpr\Gm@tmargin-\headheight-10mm,
          footskip = \dimexpr\Gm@bmargin-10mm,
          %showframe,
          bindingoffset=0mm}
\setlength{\footheight}{\baselineskip}
\makeatother

%! Versuch von besseren Seitenumbrüchen
\clubpenalty = 10000
\widowpenalty = 10000
\displaywidowpenalty = 10000
\widowpenalties= 3 10000 10000 150

\linespread{1.3}
\newcommand\frontmatter{%
    \cleardoublepage
  %\@mainmatterfalse
  \pagenumbering{Roman}}

\newcommand\mainmatter{%
    \cleardoublepage
 % \@mainmattertrue
  \pagenumbering{arabic}}

\newcommand\backmatter{%
  \if@openright
    \cleardoublepage
  \else
    \clearpage
  \fi
 % \@mainmatterfalse
   }

%! INHALTSVERZEICHNIS / ANHANGSSVERZEICHNIS
\DeclareNewTOC[%
  %owner=\jobname,
  tocentrystyle=tocline,
  tocentryentrynumberformat=\PrefixBy{Anhang},
  listname={Anhangverzeichnis}% Titel des Verzeichnisses
]{atoc}% Dateierweiterung (a=appendix, toc=table of contents)


% ? Ab hier beginnen eigene Befehle um den Umgang zu erleichtern   
\newcommand{\logisch}[1]{$``#1``$}
\newcommand{\bild}[4][1.0]{\begin{figure}[H]
                      \centering
                      \includegraphics[width=#1\columnwidth]{bilder/#2}
                      \caption{#3}
                      \label{fig:#4}
                      \end{figure}}
\newcommand{\fullref}[1]{(siehe Kapitel~\ref{#1}~-~\nameref{#1})}
\newcommand{\literef}[1]{Kapitel~\ref{#1}~-~\nameref{#1}}
\newcommand{\aref}[1]{\emph{\hyperref[{#1}]{Anhang \ref{#1}}}}
\newcommand{\bref}[1]{\emph{\hyperref[{fig:#1}]{Abbildung \ref{fig:#1}}}}
\newcommand{\striche}[1]{\glqq #1\grqq{}}
\newcommand{\vglink}[2]{\footnote{\hspace{0.5em}vgl.~\href{#1}{#1}~(#2)}}
\newcommand{\python}[1]{\mintinline{python}{#1}}
\newcommand{\svg}[4][1.0]{\begin{figure}[H]
  \centering
  \includesvg[width=#1\columnwidth,inkscapelatex=false]{bilder/#2}
  \caption{#3}
  \label{fig:#4}
  \end{figure}}
\newcommand{\fn}[1]{\footnote{\hspace{0.5em}#1}}
