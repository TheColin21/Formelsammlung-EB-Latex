%! LaTeX Vorlage
\documentclass[12pt, fleqn, captions=nooneline, titlepage, footsepline, headsepline, toc=sectionentrywithdots, listof=entryprefix, bibliography=totoc, parskip=half-]{scrartcl}

\usepackage{silence} %unnötige Warnungen unterdrücken
\WarningFilter{latex}{You have requested}
\WarningFilter{scrlayer-scrpage}{\headheight to low}
\WarningFilter{scrlayer-scrpage}{\footheight to low}
\WarningFilter{scrlayer-scrpage}{Very small head height detected}
\WarningFilter{fvextra}{}
\WarningFilter{lineno}{}

\usepackage{tocbasic}
\usepackage[ngerman]{babel}
%\usepackage[backend=biber, style=authortitle, isbn=false]{biblatex}

%! Das Inhaltsverzeichnis wird an dieser Stelle formatiert.
\RedeclareSectionCommands[beforeskip=-.1\baselineskip, afterskip=.1\baselineskip, tocindent=0pt, tocnumwidth=45pt]{section,subsection,subsubsection}

%! Formatierung aller Verzeichnisse
\renewcaptionname{ngerman}{\refname}{Quellenverzeichnis}
%\setuptoc{toc}{totoc}
\setuptoc{lof}{totoc}
\setuptoc{lot}{totoc}
\renewcommand*\listoflofentryname{\bfseries\figurename}
\BeforeStartingTOC[lof]{\renewcommand*\autodot{\space\space\space\space}}
\addtokomafont{captionlabel}{\bfseries}
\renewcommand*\listoflotentryname{\bfseries\tablename}
\BeforeStartingTOC[lot]{\renewcommand*\autodot{\space\space\space\space}}

%? Anhangsverzeichnis
\providecaptionname{ngerman}{\listofatocentryname}{Anhang}
\DeclareTOCStyleEntry[level=0, indent=50pt, numwidth=5cm, entrynumberformat={Anlage}]{tocline}{atoc}

\makeatletter
\AfterTOCHead[atoc]{\let\if@dynlist\if@tocleft}
\newcommand*{\useappendixtocs}{
  \renewcommand*{\ext@toc}{atoc}
  \RedeclareSectionCommands[tocindent=0pt]{section, subsection, subsubsection}
  \RedeclareSectionCommands[tocnumwidth=85pt]{section, subsection, subsubsection}
  \addtokomafont{sectionentry}{\mdseries}
  \renewcommand*\listoflofentryname{\mdseries}
  \renewcommand{\thesection}{\arabic{section}}
  \renewcommand{\@dotsep}{10000}
  }
\newcommand*{\usestandardtocs}{
  \renewcommand*{\ext@toc}{toc}
  }
\makeatother

%! Ermöglicht die Ausgabe des aktuellen Titels
\usepackage{nameref}
\makeatletter
\newcommand*{\currentname}{\@currentlabelname}
\makeatother
\usepackage{xcolor}


\usepackage[normalem]{ulem}
\useunder{\uline}{\ul}{}
%! Latex-Packages, die verwendet werden
\usepackage{amsmath}
\usepackage{amssymb}
\usepackage{amsthm}
\usepackage{tabularx}
\usepackage{multirow}
\usepackage{setspace} 
\usepackage{booktabs}
\usepackage{svg}
\usepackage{graphicx}
\usepackage{float}
\usepackage[a4paper,lmargin={2.5cm},rmargin={2.5cm},tmargin={2cm},bmargin={2cm}]{geometry}
\usepackage{lineno}
\usepackage{csquotes}
\usepackage[european,straightvoltages]{circuitikz}
\usepackage{icomma}
\usepackage[normalem]{ulem}
\usepackage{mathtools}
\useunder{\uline}{\ul}{}

%! Code Integration im Dokument
%? Inklusive Erzeugung eines Custom Enviroments für Programmcodes
\usepackage[outputdir=latex.out]{minted}
%\SetupFloatingEnvironment{listing}{name=Program code}
%\SetupFloatingEnvironment{listing}{listname=List of Program Code}


\DeclareNewTOC[
  type=code,                         % Name der Umgebung
  types=codes,                       % Erweiterung (\listofschemes)
  float,                               % soll gleiten
  floatpos=H,
  tocentryentrynumberformat=\bfseries,                        % voreingestellte Gleitparameter
  name=Code,                         % Name in Überschriften
  listname={Programmcodeverzeichnis}, % Listenname
  % counterwithin=chapter
]{loc}
\setuptoc{loc}{totoc}

\renewcommand*{\codeformat}{%
  \codename~\thecode%
  \autodot{\space\space\space}
}
\BeforeStartingTOC[loc]{\renewcommand*\autodot{\space\space\space\space}}


%! Lang ersehntes Formelumgebung / Formelverzeichnis
\DeclareNewTOC[
  type=formel,                         % Name der Umgebung
  types=formeln,                       % Erweiterung (\listofschemes)
  float,                               % soll gleiten
  tocentryentrynumberformat=\bfseries,                        % voreingestellte Gleitparameter
  name=Formel,                         % Name in Überschriften
  listname={Formelverzeichnis}, % Listenname
  % counterwithin=chapter
]{lom}
\setuptoc{lom}{totoc}
\renewcommand*{\formelformat}{%
  \formelname~\theformel%
  \autodot{\space\space\space}
}
\BeforeStartingTOC[lom]{\renewcommand*\autodot{\space\space\space\space}}


%! Schriftart
\usepackage{helvet}
\usepackage{microtype}
\renewcommand{\familydefault}{\sfdefault}
%! Hyperref und PDF-meta
\usepackage[hidelinks]{hyperref}
\usepackage[numbered]{bookmark}
\usepackage{acronym}
\usepackage{enumitem}

%Kein Umbruch auf der titelseite
\let\endtitlepage\relax

\renewcommand*{\aclabelfont}[1]{\acsfont{#1}} %Abkürzungsverzeichnis - Formatierung
 %zB zum korrekten anzeigen von BASH-Befehlen


%! Caption um Tabellen und Abbildungen richtig zu beschriften
\usepackage{caption}
% Captions linksbündig auch wenn einzeilig
\captionsetup{
  labelsep=none,
  justification=raggedright,
  singlelinecheck=false
}
\renewcommand*{\figureformat}{%
  \figurename~\thefigure%
  \autodot{\space\space\space}
}
\renewcommand*{\tableformat}{%
  \tablename~\thetable%
  \autodot{\space\space\space}
}


%! Formatierung der Fußnotenzitate / Literaturverzeichnis


%! Kopf- und Fußzeile
\usepackage[automark]{scrlayer-scrpage} 
\pagestyle{scrheadings} 
\clearmainofpairofpagestyles 
\clearplainofpairofpagestyles
\renewcommand*{\sectionmarkformat}{}
\rohead{\textnormal{\headmark}}
%\lohead{\includegraphics[height=8mm]{bilder/logo.png}}
\lohead{}
\lofoot{}
\cofoot{} 
\rofoot{\textnormal{Seite}~\pagemark}
\makeatletter
\usepackage{geometry}
\setlength{\footheight}{18.85002pt}
\geometry{a4paper,
          left=25mm,right=25mm,top=20mm,bottom=20mm,
          includehead=false,
          includefoot=false,
          headheight = \baselineskip,
          headsep = \dimexpr\Gm@tmargin-\headheight-10mm,
          footskip = \dimexpr\Gm@bmargin-10mm,
          %showframe,
          bindingoffset=0mm}
\setlength{\footheight}{\baselineskip}
\makeatother

%! Versuch von besseren Seitenumbrüchen
\clubpenalty = 10000
\widowpenalty = 10000
\displaywidowpenalty = 10000
\widowpenalties= 3 10000 10000 150

\linespread{1.3}
\newcommand\frontmatter{%
    \cleardoublepage
  %\@mainmatterfalse
  \pagenumbering{Roman}}

\newcommand\mainmatter{%
    \cleardoublepage
 % \@mainmattertrue
  \pagenumbering{arabic}}

\newcommand\backmatter{%
  \if@openright
    \cleardoublepage
  \else
    \clearpage
  \fi
 % \@mainmatterfalse
   }

%! INHALTSVERZEICHNIS / ANHANGSSVERZEICHNIS
\DeclareNewTOC[%
  %owner=\jobname,
  tocentrystyle=tocline,
  tocentryentrynumberformat=\PrefixBy{Anhang},
  listname={Anhangverzeichnis}% Titel des Verzeichnisses
]{atoc}% Dateierweiterung (a=appendix, toc=table of contents)


% ? Ab hier beginnen eigene Befehle um den Umgang zu erleichtern   
\newcommand{\logisch}[1]{$``#1``$}
\newcommand{\bild}[4][1.0]{\begin{figure}[H]
                      \centering
                      \includegraphics[width=#1\columnwidth]{bilder/#2}
                      %\caption{#3}
                      \label{fig:#4}
                      \end{figure}}
\newcommand{\fullref}[1]{(siehe Kapitel~\ref{#1}~-~\nameref{#1})}
\newcommand{\striche}[1]{\glqq #1\grqq{}}
\newcommand{\vglink}[2]{\footnote{\hspace{0.5em}vgl.~\href{#1}{#1}~(#2)}}
\newcommand{\python}[1]{\mintinline{python}{#1}}

%Hbox warnungen ignorieren, da diese hier unvermeidlich sind
\hbadness=10000
\hfuzz=10pt
\pdfsuppresswarningpagegroup=1
\newcommand*{\bfrac}[2]{\genfrac{}{}{0pt}{}{#1}{#2}}
\setlength{\extrarowheight}{2pt}

%TODO @willy (wenn ich lust hab) kevins formelsammlung + mein gekritzel + reinhold-dokumente durchgehen und alle formeln doublechecken ~w
\begin{document}
\frontmatter
\title{Formelsammlung Elektronische Bauelemente}
%\author{Willy Hille, Colin Pohle nach Original von Marcel Ott}
\author{Martin Dietschreit, Willy Hille, Julia Nützel,\\
Colin Pohle, Eric Richter, Kevin Wenke\\
nach Original von Marcel Ott.}
\maketitle
\vfill
\scriptsize
Diese Formelsammlung basiert auf der Light-Version einer \href{https://github.com/DSczyrba/Vorlage-Latex/}{Latex-Vorlage} von \href{https://github.com/jemand771}{Willy Hille}, \href{https://github.com/TheColin21}{Colin Pohle} und \href{https://github.com/DSczyrba}{Dominic Sczyrba}.
\vfuzz=10pt
\tableofcontents
\vfuzz=0.1pt

\mainmatter
\section{E-Reihen}
    \renewcommand{\arraystretch}{0.65}
\begin{table}[H]
\footnotesize
\centering
\begin{tabular}{|c|c|c|c|c|c|c|c|c|c|c|}
%formatierung leicht geändert, understriche sind mä, kopfzeilen auf fußzeilen gespiegelt
\cline{1-5} \cline{7-11}
{\textbf{E6}}                   & {\textbf{E12}}   & {\textbf{E24}}   & {\textbf{E48}}    & {\textbf{E96}} & $\quad\quad\quad$ & {\textbf{E6}}                  & {\textbf{E12}}    & {\textbf{E24}}   & {\textbf{E48}}    & {\textbf{E96}} \\ \cline{1-5} \cline{7-11} 
\multicolumn{1}{|c|}{\textit{20\%}} & \textit{10\%}        & \textit{5\%}         & \textit{2\%}          & \textit{1\%}       &         & \multicolumn{1}{c|}{\textit{20\%}} & \textit{10\%}         & \textit{5\%}         & \textit{2\%}          & \textit{1\%}       \\ \cline{1-5} \cline{7-11} 
\multirow{16}{*}{1}                 & \multirow{8}{*}{1}   & \multirow{4}{*}{1}   & \multirow{2}{*}{1}    & 1                  &         & \multirow{18}{*}{3,3}              & \multirow{10}{*}{3,3} & \multirow{6}{*}{3,3} & \multirow{2}{*}{3,16} & 3,16               \\ \cline{5-5} \cline{11-11} 
                                    &                      &                      &                       & 1,02               &         &                                    &                       &                      &                       & 3,24               \\ \cline{4-5} \cline{10-11} 
                                    &                      &                      & \multirow{2}{*}{1,05} & 1,05               &         &                                    &                       &                      & \multirow{2}{*}{3,32} & 3,32               \\ \cline{5-5} \cline{11-11} 
                                    &                      &                      &                       & 1,07               &         &                                    &                       &                      &                       & 3,4                \\ \cline{3-5} \cline{10-11} 
                                    &                      & \multirow{4}{*}{1,1} & \multirow{2}{*}{1,1}  & 1,1                &         &                                    &                       &                      & \multirow{2}{*}{3,48} & 3,48               \\ \cline{5-5} \cline{11-11} 
                                    &                      &                      &                       & 1,13               &         &                                    &                       &                      &                       & 3,57               \\ \cline{4-5} \cline{9-11} 
                                    &                      &                      & \multirow{2}{*}{1,15} & 1,15               &         &                                    &                       & \multirow{4}{*}{3,6} & \multirow{2}{*}{3,65} & 3,65               \\ \cline{5-5} \cline{11-11} 
                                    &                      &                      &                       & 1,18               &         &                                    &                       &                      &                       & 3,74               \\ \cline{2-5} \cline{10-11} 
                                    & \multirow{8}{*}{1,2} & \multirow{4}{*}{1,2} & \multirow{2}{*}{1,21} & 1,21               &         &                                    &                       &                      & \multirow{2}{*}{3,83} & 3,83               \\ \cline{5-5} \cline{11-11} 
                                    &                      &                      &                       & 1,24               &         &                                    &                       &                      &                       & 3,92               \\ \cline{4-5} \cline{8-11} 
                                    &                      &                      & \multirow{2}{*}{1,27} & 1,27               &         &                                    & \multirow{8}{*}{3,9}  & \multirow{4}{*}{3,9} & \multirow{2}{*}{4,02} & 4,02               \\ \cline{5-5} \cline{11-11} 
                                    &                      &                      &                       & 1,3                &         &                                    &                       &                      &                       & 4,12               \\ \cline{3-5} \cline{10-11} 
                                    &                      & \multirow{4}{*}{1,3} & \multirow{2}{*}{1,33} & 1,33               &         &                                    &                       &                      & \multirow{2}{*}{4,22} & 4,22               \\ \cline{5-5} \cline{11-11} 
                                    &                      &                      &                       & 1,37               &         &                                    &                       &                      &                       & 4,32               \\ \cline{4-5} \cline{9-11} 
                                    &                      &                      & \multirow{2}{*}{1,4}  & 1,4                &         &                                    &                       & \multirow{4}{*}{4,3} & \multirow{2}{*}{4,42} & 4,42               \\ \cline{5-5} \cline{11-11} 
                                    &                      &                      &                       & 1,43               &         &                                    &                       &                      &                       & 4,53               \\ \cline{1-5} \cline{10-11} 
\multirow{16}{*}{1,5}               & \multirow{8}{*}{1,5} & \multirow{4}{*}{1,5} & \multirow{2}{*}{1,47} & 1,47               &         &                                    &                       &                      & \multirow{2}{*}{4,64} & 4,64               \\ \cline{5-5} \cline{11-11} 
                                    &                      &                      &                       & 1,5                &         &                                    &                       &                      &                       & 4,75               \\ \cline{4-5} \cline{7-11} 
                                    &                      &                      & \multirow{2}{*}{1,54} & 1,54               &         & \multirow{16}{*}{4,7}              & \multirow{8}{*}{4,7}  & \multirow{4}{*}{4,7} & \multirow{2}{*}{4,87} & 4,87               \\ \cline{5-5} \cline{11-11} 
                                    &                      &                      &                       & 1,58               &         &                                    &                       &                      &                       & 4,99               \\ \cline{3-5} \cline{10-11} 
                                    &                      & \multirow{4}{*}{1,6} & \multirow{2}{*}{1,62} & 1,62               &         &                                    &                       &                      & \multirow{2}{*}{5,11} & 5,11               \\ \cline{5-5} \cline{11-11} 
                                    &                      &                      &                       & 1,65               &         &                                    &                       &                      &                       & 5,23               \\ \cline{4-5} \cline{9-11} 
                                    &                      &                      & \multirow{2}{*}{1,69} & 1,69               &         &                                    &                       & \multirow{4}{*}{5,1} & \multirow{2}{*}{5,36} & 5,36               \\ \cline{5-5} \cline{11-11} 
                                    &                      &                      &                       & 1,74               &         &                                    &                       &                      &                       & 5,49               \\ \cline{2-5} \cline{10-11} 
                                    & \multirow{8}{*}{1,8} & \multirow{4}{*}{1,8} & \multirow{2}{*}{1,78} & 1,78               &         &                                    &                       &                      & \multirow{2}{*}{5,62} & 5,62               \\ \cline{5-5} \cline{11-11} 
                                    &                      &                      &                       & 1,82               &         &                                    &                       &                      &                       & 5,76               \\ \cline{4-5} \cline{8-11} 
                                    &                      &                      & \multirow{2}{*}{1,87} & 1,87               &         &                                    & \multirow{8}{*}{5,6}  & \multirow{4}{*}{5,6} & \multirow{2}{*}{5,9}  & 5,9                \\ \cline{5-5} \cline{11-11} 
                                    &                      &                      &                       & 1,91               &         &                                    &                       &                      &                       & 6,04               \\ \cline{3-5} \cline{10-11} 
                                    &                      & \multirow{4}{*}{2}   & \multirow{2}{*}{1,96} & 1,96               &         &                                    &                       &                      & \multirow{2}{*}{6,19} & 6,19               \\ \cline{5-5} \cline{11-11} 
                                    &                      &                      &                       & 2                  &         &                                    &                       &                      &                       & 6,34               \\ \cline{4-5} \cline{9-11} 
                                    &                      &                      & \multirow{2}{*}{2,05} & 2,05               &         &                                    &                       & \multirow{4}{*}{6,2} & \multirow{2}{*}{6,49} & 6,49               \\ \cline{5-5} \cline{11-11} 
                                    &                      &                      &                       & 2,1                &         &                                    &                       &                      &                       & 6,65               \\ \cline{1-5} \cline{10-11} 
\multirow{16}{*}{2,2}               & \multirow{8}{*}{2,2} & \multirow{4}{*}{2,2} & \multirow{2}{*}{2,15} & 2,15               &         &                                    &                       &                      & \multirow{2}{*}{6,81} & 6,81               \\ \cline{5-5} \cline{11-11} 
                                    &                      &                      &                       & 2,21               &         &                                    &                       &                      &                       & 6,98               \\ \cline{4-5} \cline{7-11} 
                                    &                      &                      & \multirow{2}{*}{2,26} & 2,26               &         & \multirow{14}{*}{6,8}              & \multirow{8}{*}{6,8}  & \multirow{4}{*}{6,8} & \multirow{2}{*}{7,15} & 7,15               \\ \cline{5-5} \cline{11-11} 
                                    &                      &                      &                       & 2,32               &         &                                    &                       &                      &                       & 7,32               \\ \cline{3-5} \cline{10-11} 
                                    &                      & \multirow{4}{*}{2,4} & \multirow{2}{*}{2,37} & 2,37               &         &                                    &                       &                      & \multirow{2}{*}{7,5}  & 7,5                \\ \cline{5-5} \cline{11-11} 
                                    &                      &                      &                       & 2,43               &         &                                    &                       &                      &                       & 7,68               \\ \cline{4-5} \cline{9-11} 
                                    &                      &                      & \multirow{2}{*}{2,49} & 2,49               &         &                                    &                       & \multirow{4}{*}{7,5} & \multirow{2}{*}{7,87} & 7,87               \\ \cline{5-5} \cline{11-11} 
                                    &                      &                      &                       & 2,55               &         &                                    &                       &                      &                       & 8,06               \\ \cline{2-5} \cline{10-11} 
                                    & \multirow{8}{*}{2,7} & \multirow{4}{*}{2,7} & \multirow{2}{*}{2,61} & 2,61               &         &                                    &                       &                      & \multirow{2}{*}{8,25} & 8,25               \\ \cline{5-5} \cline{11-11} 
                                    &                      &                      &                       & 2,67               &         &                                    &                       &                      &                       & 8,45               \\ \cline{4-5} \cline{8-11} 
                                    &                      &                      & \multirow{2}{*}{2,74} & 2,74               &         &                                    & \multirow{6}{*}{8,2}  & \multirow{4}{*}{8,2} & \multirow{2}{*}{8,66} & 8,66               \\ \cline{5-5} \cline{11-11} 
                                    &                      &                      &                       & 2,8                &         &                                    &                       &                      &                       & 8,87               \\ \cline{3-5} \cline{10-11} 
                                    &                      & \multirow{4}{*}{3}   & \multirow{2}{*}{2,87} & 2,87               &         &                                    &                       &                      & \multirow{2}{*}{9,09} & 9,09               \\ \cline{5-5} \cline{11-11} 
                                    &                      &                      &                       & 2,94               &         &                                    &                       &                      &                       & 9,31               \\ \cline{4-5} \cline{9-11} 
                                    &                      &                      & \multirow{2}{*}{3,01} & 3,01               &         &                                    &                       & \multirow{2}{*}{9,1} & \multirow{2}{*}{9,53} & 9,53               \\ \cline{5-5} \cline{11-11} 
                                    &                      &                      &                       & 3,09               &         &                                    &                       &                      &                       & 9,76               \\ \cline{1-5} \cline{7-11} 
\multicolumn{1}{|c|}{\textit{20\%}} & \textit{10\%}        & \textit{5\%}         & \textit{2\%}          & \textit{1\%}       &         & \multicolumn{1}{c|}{\textit{20\%}} & \textit{10\%}         & \textit{5\%}         & \textit{2\%}          & \textit{1\%}       \\ \cline{1-5} \cline{7-11}
\cline{1-5} \cline{7-11}
{\textbf{E6}}                   & {\textbf{E12}}   & {\textbf{E24}}   & {\textbf{E48}}    & {\textbf{E96}} & $\quad\quad\quad$ & {\textbf{E6}}                  & {\textbf{E12}}    & {\textbf{E24}}   & {\textbf{E48}}    & {\textbf{E96}} \\ \cline{1-5} \cline{7-11}
\end{tabular}
\end{table}
\noindent
\small
Verwendung: Wert aus der gewünschten Reihe auswählen (Standard: \textbf{E24}) und mit $10^n$ für beliebiges ganzzahliges $n$ multiplizieren. Der letzte Eintrag in \textbf{E24} kann also $9,1\Omega$, $910\Omega$, aber auch $0,00091\Omega$ beschreiben. (Letzterer hat evtl. keinen praktischen Nutzen.)\\
\normalsize
$C_{gew}\geq C_{berechnet}\quad\quad R_{\min} \leq R_{gew}\pm R_{gew}\cdot Toleranz\leq R_{\max}$
\section{Grundlagen ET}
    \begin{table}[h!]
\renewcommand{\arraystretch}{1.2}
\centering
\begin{tabular}{|l|l|l|l|}
\hline
Elektrischer Parameter & Einheit & Symbol          & Definition                                                                                                       \\ \hline
Spannung               & Volt    & $[U] = 1V$      & $1V = 1\frac{J}{C} = 1\frac{Nm}{As} = 1\frac{kg\cdot{}m^2}{A\cdot{}s^3} = 1A\cdot\Omega$                         \\ \hline
Strom                  & Ampere  & $[I] = 1A$      & $1A = 1\frac{C}{s} = 1\frac{V}{\Omega}$                                                                          \\ \hline
Widerstand             & Ohm     & $[R] = 1\Omega$ & $R = \frac{U}{I} = \frac{1}{G}$                                                                                  \\ \hline
Leitfähigkeit          & Siemens & $[G] = 1 S$     & $G = \frac{1}{R}$                                                                                                \\ \hline
Kapazität              & Farad   & $[C] = 1F$      & $C = \frac{Q}{U}\quad 1F = 1\frac{As}{V}$\\ \hline
Ladung                 & Coulomb & $[Q] = 1C$      & $1C = 1As$                                                                                                       \\ \hline
Induktivität           & Henry   & $[L] = 1H$      & $1H = 1\frac{Vs}{A}$                                                                                             \\ \hline
Leistung               & Watt    & $[P] = 1W$      & $P = U\cdot I=\frac{U^2}{R}=I^2\cdot R$\\ \hline
Frequenz               & Hertz   & $[f] = 1Hz$     & $f = \frac{1}{T}$                                                                                                \\ \hline
\end{tabular}
\end{table}
    Hinweis: $U_{a-}$ bzw. $U_{a+}$ ist nur kürzere Schreibweise für $U_{a_{\min}}$ bzw. $U_{a_{\max}}$
\section{dB \& dB-Rechnung} %nach Kevin Formelsammlung
    $V_P=\frac{P_{aus}}{P_{ein}}\quad\quad 
    V_U=\frac{U_{aus}}{U_{ein}}\quad\quad
    V_{P_{dB}}=10\lg\frac{P_{aus}}{P_{ein}}\quad\quad
    V_{U_{dB}}=20\lg\frac{U_{aus}}{U_{ein}}$ \\
    negativer dB-Wert: Dämpfung (z.B. $20dB\rightarrow $x$100; -30dB\rightarrow $x$\frac{1}{1000}$)\\
    Stromverstärkung (ungewöhnlich) verhält sich wie Spannungsverstärkung\\

    \begin{minipage}{0.5\columnwidth}
        \renewcommand{\arraystretch}{0.9}
        \begin{table}[H]
            \centering
            \begin{tabular}{|l|l|l|}
                \hline
                $dB$  & $V_P$                 & $V_U$                  \\ \hline
                $0$   & $1$                   & $1$                    \\ \hline
                $1$   & $\sqrt[10]{10}=1,26$  & $\sqrt[20]{10}=1,12$   \\ \hline
                $2$   & $\sqrt[5]{10}=1,58$   & $\sqrt[10]{10}=1,26$   \\ \hline
                $3$   & $2$                   & $\sqrt{2}=1,41$        \\ \hline
                $6$   & $4$                   & $2$                    \\ \hline
                $10$  & $10$                  & $\sqrt{10}=3,16$       \\ \hline
                $20$  & $100$                 & $10$                   \\ \hline
                $40$  & $10^4$                & $100$                  \\ \hline
                $60$  & $10^6$                & $1000$                 \\ \hline
                $80$  & $10^8$                & $10^4$                 \\ \hline
                $100$ & $10^{10}$             & $10^5$                 \\ \hline
            \end{tabular}
        \end{table}
    \end{minipage}
    \begin{minipage}{0.5\columnwidth}
        \textbf{bezogene dB-Werte:}
        \renewcommand{\arraystretch}{0.9}
        \begin{table}[H]
            \centering
            \begin{tabular}{|l|l|l|}
                \hline
                Größe          & Bezugswert & Einheit    \\ \hline
                Leistung $P_0$ & $1mW$      & $dBm$      \\ \hline
                Leistung $P_0$ & $1W$       & $dBW$      \\ \hline
                Spannung $U_0$ & $1\mu V$   & $dB\mu V$  \\ \hline
                Spannung $U_0$ & $1V$       & $dBV$      \\ \hline
                Spannung $U_0$ & $774,6mV$  & $dBu$      \\ \hline
                Strom $I_0$    & $1A$       & $dBA$      \\ \hline
                Strom $I_0$    & $1,291mA$  & $dBi$      \\ \hline
            \end{tabular}
        \end{table}
        $dBm$: $V_{P_{dB}}=10\lg\frac{P_{aus}}{1mW}$\\
        $db\mu$: $V_{U_{dB}}=20\lg\frac{U_{aus}}{1\mu V}$
    \end{minipage}

    $\longrightarrow$ Zerlegen\\
    $V_{ges}=\prod\limits_{i=1}^n V_i=V_1\cdot V_2\cdot V_3\cdot\ \dots\ \cdot V_n\\
    V_{ges_{dB}}=\sum\limits_{i=1}^n V_i=V_{1_{dB}}+V_{2_{dB}}+V_{3_{dB}}+\ \dots\ +V_{n_{dB}}$\\

\section{Halbleiter allgemein}\label{sec:halbleiter}
    \begin{table}[H]
    \begin{tabularx}{\columnwidth}{|X|X|}
        \hline
        \multicolumn{2}{|c|}{\textbf{Besetzungswahrscheinlichkeit:}}\\
        \hline
        $P(W)=\frac{1}{1+e^{\frac{W-W_F}{k\cdot T}}}$ & $k=1,38\cdot 10^{-23}\frac{Ws}{K}$ \\
        & $T=$ absolute Temperatur \\
        & $W_F=$ Fermi-Nivaeu \\
        \hline
        \multicolumn{1}{|c|}{$T=0$}                   & \multicolumn{1}{c|}{$T>0$} \\
        $W>W_F\rightarrow 0$                          & $P(W_F)=\frac{1}{2}$\\
        $W<W_F\rightarrow 1$                          & \\
        \hline
    \end{tabularx}
\end{table}
\begin{table}[H]
    \begin{tabularx}{\textwidth}{|X X|}
        \hline
        \multicolumn{2}{|c|}{\textbf{Legende:}}\\
        \hline
        $I_F=$Effektivwert                      & $I_{FM}=$periodischer Spitzenstrom \\
        $I_{FS}=$Stoßspitzenstrom               & $I_0=$Richtstrom (arithm. Mittel)\\
        $t_i=$ Einschaltdauer                   & $T=$ Gesamtdauer \\
        \multicolumn{2}{|c|}{$g=\frac{t_i}{T}$}\\
        $U_{RRM}=$period. Spitzensperrspannung  & $U_{RSM}=$eindeut. Spitzensperrspannung \\
        $P_{tot}=$tot. Verlustleistung          & $P_V=U_F\cdot I_F + U_R\cdot I_R$ \\
        & $P_V\leq P_{tot}$ \\
        \multicolumn{2}{|c|}{$R_{th JU}=$therm. Widerstand zw. Sperrschicht \& Umgebung}\\
        \multicolumn{2}{|c|}{$R_{th JU}=\frac{\vartheta_J-\vartheta_U}{P_V} \bfrac{\vartheta_J\ Temperatur\ Sperrschicht}{\vartheta_U\ Temperatur\ Umgebung}\quad
        R_{th_{JU}}=R_{th_{JG}}+R_{th_{GK}}+R_{th_{K}}\quad [R_{th}]=\frac{K}{W}$}\\
        \multicolumn{2}{|c|}{$P_{V_{\max}}=\frac{\vartheta_{I_{\max}}-\vartheta_U}{R_{th_{JG}}+R_{th_{GK}}+R_{th_{K}}}$}\\
        $R_F=$Gleichstromwiderstand Diode       & $R_R=$stat. Sperrwiderstand \\
        $r_F=$dyn. Durchlasswiderstand          & $C_D=$Diodenkapazität \\
        $R=\frac{U}{I}$                         & $r_f=\frac{\Delta U_F}{\Delta I_F}$ \\
        $t_{rr}=$Sperrverzögerungszeit          & \\
        \hline
    \end{tabularx}
\end{table}
\svg[1.0]{durchlassstrom}{Durchlassstrom}{durchlassstrom}
\section{Einweggleichrichter}\label{sec:Einweggleichrichter}
    \begin{circuitikz}
\draw % transformator
(1,0) node[transformer, anchor=A1](T){}
(T.A1) node[anchor=east] {}
(T.A2) node[anchor=east] {}
(T.B1) node[anchor=west, label=below:+/-] {}
(T.B2) node[anchor=west, label=above:-/+] {}
% L1
(0,0) node[label={left:L1}](L1){}
(L1) to [short, o-] (T.A1)
% N
(T.A2) to [short, -o] ++(-1,0)
++(0,0) node[label={left:N}](N){}
% U
(L1) to [open, v=$U$] (N)
% a
(T.B1) to [short, -o, i=$i_1$] ++(2,0)
++(0,0) node[label={above:a}](a){}
% b
(T.B2) to [short, -o] ++(2, 0)
++(0,0) node[label={below:b}](b){}
% diode
(a) to [diode, -*, v=$u_D$] ++(2,0)
++(0,0) node[](cla){}
to [short, -o] ++(2,0)
++(0,0) node[](u2a){}
to [short, i=$i_{Entladung}$] ++(1,0)
% go down, R
to [R, i=$i_L$, R=$R_L$] ++(0,-2|-T.B2)
% left again
to [short, -o] ++(-1,0)
% u2
++(0,0) node[](u2b){}
(u2a) to [open, v=$u_2$] (u2b)
% back to b
to [short, -*] ++(-2,0)
++(0,0) node[](clb){}
to [short, i=$i_{Aufladung}$] (b)
% CL and u1
(cla) to [capacitor, C=$C_L$] (clb)
(a) to [open, v=$u_1$] (b)
% kondensator-ströme
(cla) to [open, i=$i_{Aufladung}$] (clb)
(clb) to [open, i=$i_{Entladung}$] (cla)
;
\end{circuitikz}

    \begin{table}[H]
        \begin{tabularx}{\columnwidth}{l l l}
            Wert            & Formel                                            & Bemerkung \\
            $C_L$           & $=10\mu F\cdot [I_L]$                             & \\
            $U_2$           & $= \sqrt{2}\cdot U_1-U_{Diode}$                   & $U_{Diode}\approx 0,6V$ im Leerlauf \\
            $U_2$           & $= 1,2\cdot U_1$                                  & Belastungsfall \\
            $I_{FM}$        & $\geq1,2\dots 5\cdot I_{L-}$                      & \\
            $U_{RM}$        & $=2\cdot \sqrt{2}\cdot U_1$                       & Ungünstigster Fall: Leerlauf \\
            $U_{RM}$        & $=3\cdot \sqrt{2}\cdot U_1$                       & praktisch sinnvolle Auswahl \\
            $I_{L-}$        & $=\frac{U_{2-}}{R_{L\,\min}}$                     & Laststrom \\
            $\ddot{U}$      & $=\frac{U_{Prim\ddot{a}r}}{U_{Sekund\ddot{a}r}}$  & Trafoübersetzung \\
            $U_e$           & $=2\dots 4\cdot U_a$                              & praktisch $2\cdot U_a$ \\
            $U_{Br}$        & $=50Hz$                                           & Deutschland \\ % mein Herz in Flammen
            $U_{Br\,eff}$   & $=4,8\cdot 10^{-3}\cdot s\cdot\frac{I_{L-}}{C_L}$ & \\
            $U_{Br\,SS}-$   & $=14\cdot 10^{-3}\cdot s\cdot\frac{I_{L-}}{C_L}$  & \\
            $W$             & $=\frac{U_{Br\,eff}}{U_{L-}}$                     & Welligkeit \\ %! in Original anders aber laut Kevin so korrekt
        \end{tabularx}
    \end{table}
\section{Zweipuls-Mittelpunkt-Schaltung (M2U)}\label{sec:zweipuls-mittelpunkt-schaltung} %! in Original fälschlicherweise "Zweipols... " genannt, hoffe ich
    \svg[1.0]{m2u}{M2U}{m2u}

    \begin{table}[H]
        \begin{tabularx}{\columnwidth}{l l l}
            Wert            & Formel                                             & Bemerkung \\
            $U_{2-}$        & $=\sqrt{2}\cdot U_1-U_F$                           & $\approx 1,3\cdot U_1 $ \\
            $I_{FM}$        & $\geq 0,72\cdot I_{L-}$                            & \\
            $U_{RM}$        & $\geq 3\cdot \sqrt{2}\cdot U_1$                    & \\
            $U_{Br\, eff}$  & $=1,8\cdot 10^{-3}\cdot s\cdot\frac{I_{L-}}{C_L}$  & \\
            $U_{Br\, SS}$   & $=7\cdot 10^{-3}\cdot s\cdot\frac{I_{L-}}{C_L}$    & \\
        \end{tabularx}
    \end{table}

\section{Brückengleichrichter = \striche{Grätzbrücke}}\label{sec:bruckengleichrichter}
    \begin{minipage}{0.5\columnwidth}
        \begin{table}[H]
            \begin{tabularx}{\columnwidth}{l l}
                Wert            & Formel \\
                $U_{2-leer}$    & $=\sqrt{2}\cdot U_1-2\cdot U_F$ \\
                $U_{2-}$        & $\approx 1,3\cdot U_1$ \\
                $I_{FM}$        & $\geq 0,72\cdot I_{L-}$ (an den Dioden) \\
                $U_{RM}$        & $\geq 1,5\cdot\sqrt{2}\cdot U_1$ (an den Dioden) \\
                $U_{Br\,eff}$   & $=1,8\cdot 10^{-3}\cdot s \cdot\frac{I_{L-}}{C_L}$ \\
                $U_{Br\,SS}$    & $=7\cdot 10^{-3}\cdot s \cdot\frac{I_{L-}}{C_L}$ \\
                $C_L$           & $=\frac{I_L}{100}\frac{F}{A}$ (Standard: $10\frac{\mu F}{mA}$)\\
            \end{tabularx}
        \end{table}
        $U_{2-}=U_2\,\text{gleichgerichtet}$ 
    \end{minipage}
    \begin{minipage}{0.5\columnwidth}
        \svg[1.0]{brueckengleichrichter}{Brückengleichrichter}{brückengleichrichter}
    \end{minipage}

\section{RC-Sieb}\label{sec:rc-sieb}
    \renewcommand{\arraystretch}{0.9}
    \begin{table}[H]
        \begin{tabularx}{\columnwidth}{l l l}
            Wert                        & Formel                                                    & Bemerkung \\
            $\frac{U_{Br2}}{x_{Cs}}$    & $=\frac{U_{Br1}}{\sqrt{R_{2_S}+x_{2_{Cs}}}}$               & Vorraussetzung: $R_s\gg x_{Cs}\to G\approx\frac{R_s}{x_{Cs}}$ \\
            $\frac{U_{Br1}}{U_{Br2}}$   & $=G$                                                      & Glättungsfaktor \\
            $G$                         & $=\sqrt{(\frac{R_s}{x_{Cs}})_2+1}$                        & sinnvoll: $10\dots 20$ \\
            $G$                         & $=2\pi\cdot f_{Br}\cdot C_s\cdot  R_s$                    & sinnvoll: $10\dots 20$ \\
            $x_{Cs}$                    & $=\frac{1}{2\pi\cdot f_{Br}\cdot C_s}$                    & \\
            $U_{A-}$                    & $=U_{2-}-U_{Rs}$                                          & \\
            $U_{Rs}$                    & $\leq 0,1\cdot U_2$                                       & \\
            $U_{Rs}$                    & $=R_S\cdot I_{L-}$                                        & \\
        \end{tabularx}
    \end{table}
    \begin{circuitikz}
\draw
% R
(0,0) node[](A1){}
to [short, o-*] ++(2,0)
++(0,0) node[](cla) {}
to [R, *-*, R=$R_S$] ++(3,0)
++(0,0) node[](csa){}
to [short, -o] ++(3,0)
++(0,0) node[](u2a){}
to [short] ++(1,0)
%down
to [R, R=$R_L$] ++(0,-2)
++(0,0) node[](lower){}
%left
to [short, -o] (lower-|u2a)
to [short, -*] (lower-|csa)
to [short, -*] (lower-|cla)
to [short, -o] (lower-|A1)
%down parallels
(cla) to [capacitor, C=$C_L$, v=$U_{Breff_1}$] (lower-|cla)
(csa) to [capacitor, C=$C_S$] (lower-|csa)
(u2a) to [open, v=$U_{Breff_1}$] (lower-|u2a)
% u3
%to [open, v=$U_{Breff_2}$] ++(0, -2)
%to [short, o-*] ++(-2,0)
;
\end{circuitikz}  

\section{LC-Sieb}\label{sec:lc-sieb}
    \begin{minipage}{0.65\columnwidth}
        \begin{table}[H]
            \begin{tabularx}{\columnwidth}{l l}
                Wert    & Formel \\
                $G$     & $=\frac{x_L}{x_C}-1$ \\
                $x_L$   & $=2\pi\cdot f_{Br}\cdot L_S$ \\
                $x_C$   & $=\frac{1}{2\pi\cdot f_{Br}\cdot C_S}$ \\
                $G$     & $=(2\pi\cdot f_{Br})_2\cdot L_S\cdot C_S$ \\
            \end{tabularx}
        \end{table}
    \end{minipage}
    \begin{minipage}{0.35\columnwidth}
        \svg[0.8]{lc-sieb}{LC-Sieb}{lc-sieb}
    \end{minipage}

\section{Z-Diode (Zener-Diode)}\label{sec:zener-diode}
    Diodenbelastung: $P_{v_{ZD}}=I_{Z_{Betrieb\max}}\cdot U_Z$\\
    Fall 1: $U_E=konst.,\, I_L\, ist\, variabel$\\
    \hspace*{1cm} ohne Last $I_L=0\: R_V=\frac{U_E-U_Z}{I_Z}$\\
    \hspace*{1cm} mit Last $R_V=\frac{U_E-U_Z}{I_Z+I_L}\: P_{tot}=I_{Z\, \max}\cdot U_Z$\\
    Fall 2: $U_E\, variabel,\, I_L=konst.$\\
    \hspace*{1cm} $U_E=U_{R_V}+U_Z=(I_Z+I_L)\cdot R_V+U_Z$\\
    \hspace*{1cm} $R_V=\frac{U_E-U_Z}{I_Z+I_L}$\\
    Fall 3: $U_E\, und\, I_L\, variabel\quad S=relativer\, Stabilisierungsfaktor$
    \begin{table}[H]
        \renewcommand{\arraystretch}{1.2}
        \begin{tabularx}{\columnwidth}{l l l}%!Dritte Spalte jeweils auf nächster Zeile, um Code nicht zu breit zu gestalten
            $I_{F\max}=\frac{P_{tot}}{U_F}$ &
            $P_{Rv}=\frac{(U_{E\max}-U_Z)^2}{R_V}$ &
            $U_E=2\cdot U_Z$
            \\
            $I_L=\frac{U_A}{R_L}=\frac{U_Z}{R_L}$ &
            $P_{RV\max}=\frac{\left(U_{e\max}-U_Z\right)^2}{R_{V_{gew}}}$ &
            $U_{R_L}=U_B-U_F$
            \\
            $I_{L_{\min}}=\frac{U_Z}{R_{L_{\max}}}$ &
            $P_{S\max}=U_{R_L}\cdot I_{F\max}P_{tot}\cdot\left(\frac{U_B}{U_F}-1\right)$ &
            $R_{V\min}=\frac{U_{E\max}-U_Z}{I_{Z\max}+I_{L\min}}$
            \\
            $I_{L_{\max}}=\frac{U_Z}{R_{L_{\min}}}$ &
            $S_{\min}=(1+\frac{R_V\,gew.\,\min}{r_z})\cdot\frac{U_Z}{U_{E\max}}$ &
            $R_{Vmax}=\frac{U_{E\min}-U_Z}{I_{Z\min}+I_{L\max}}$
            \\
            $I_{Z_{\min}}=0,1\cdot I_{Z_{\max}}$ &
            $S_{\max}=\left(1+\frac{R_{V\,gew\max}}{r_z}\right)\cdot\frac{U_a}{U_{e\min}}$ &
            $R_{Vgew\min}=\frac{U{e_{\max}}-U_Z}{I_{ZBetr_{\max}}+I_{L\min}}$
            \\
            $I_{Z\max}=\frac{P_{tot}}{U_Z}$ &
            $S=(1+\frac{R_V}{r_z})\cdot\frac{U_Z}{U_e}$ &
            $R_{L\min}=\frac{\left(U_b-U_F\right)^2}{P_{S\max}}$
            \\
            \multicolumn{2}{l}{$I_{Z{Betrieb_{\max}}}=\frac{U{e_{\max}}-U_Z}{R_{Vgew\min}}-I_{L\min}$} &
            $R_{V_{\min}}\leq R_{V_{gew}} \leq R_{V_{\max}}$
            \\
            $R_{V_{gew}}$ meist gegen $R_{V_{\max}}$
        \end{tabularx}
    \end{table}
    Leistung über Diode auch mit berechnen, wenn das nicht explizit gefragt ist.\\
    \begin{circuitikz}[scale=2]
\draw
(0,-2) to [zDo, *-*, i<=$I_Z$] ++(0,2)
to [R, o-*, R=$R_V$] ++(-2,0)
to [open, o-o, v=$U_E$] ++(0,-2)
to [short, -*] ++(2,0)
to [short, *-] ++(2,0)
to [R, R=$R_L$, v<=$U_a$] ++(0,2)
to [short, *-, i<=$I_L$] ++(-2,0)
;
\end{circuitikz}

\section{Spannungsbegrenzung}\label{sec:spannungsbegrenzung}
    a) Spannungsgrenzen mit Z-Dioden\\
    \begin{minipage}{0.4\columnwidth}
        $pos.\, HW\: U_a=U_S+U_{Z2}$\\
        $neg.\, HW\: U_a=U_S+U_{Z1}$
    \end{minipage}
    \begin{minipage}{0.4\columnwidth}
        \svg[0.7]{spannungsgrenze_z}{Spannungsgrenze Z-Dioden}{spannung_z}
    \end{minipage}\\
    b) Spannungsgrenzen mit Gegenspannung\\
    \begin{minipage}{0.4\columnwidth}
        $U_a=U_S+U_V$\\
        $U_e>U_S+U_V$
    \end{minipage}
    \begin{minipage}{0.4\columnwidth}
        \svg[0.7]{spannungsgrenze_gegenspannung}{Spannungsgrenze Gegenspannung}{spannung_gegen}
    \end{minipage}\\
    c) Spannungsgrenzen mit Klipperschaltung\\
    \begin{minipage}{0.4\columnwidth}
        $U_a=U_r$\\
        $U_e=U_{S1}+U_{SL}+U_R$\\
        $U_e<2\cdot U_s$
    \end{minipage}
    \begin{minipage}{0.4\columnwidth}
        \svg[0.7]{spannungsgrenze_klipper}{Spannungsgrenze Klipper}{spannung_klipper}
    \end{minipage}\\
    
\section{Lineare Spannungsregler}
    \begin{minipage}{0.4\columnwidth}    
        \subsection{Festspannungs-Regler}
            \svg{festspannungs-regler}{Festspannungsregler}{festspannungsregler}
    \end{minipage}
    \begin{minipage}{0.6\columnwidth}    
        \subsection{Spannungsregler mit einstellbarer Ausgangsspannung}
            \svg[0.7]{variabler-spannungsregler}{einstellbarer Spannungsregler}{variabler-spannungsregler}
            \begin{flushright}
                $U_{Ref}=1,25V$\\
                $U_A=U_{Ref}\cdot (1+\frac{R_2}{R_1})+R_2\cdot I_{Adj} \geq 1,25V$    
            \end{flushright}
    \end{minipage}
\section{Diode als Schalter}
    \begin{minipage}{0.2\columnwidth}
        offener Schalter:\\
        $U_B=U_{Schalter}$\\
        $U_{RL}=0$
    \end{minipage}
    \begin{minipage}{0.5\columnwidth}
        \svg[1.0]{diode_schalter_auf}{Diode Schalter auf}{diode_schalter_auf}
    \end{minipage}
    \begin{minipage}{0.3\columnwidth}
        $U_{RL}=I_R\cdot R_L$\\
        $U_{Diode}=U_R\approx U_B$\\
        $Gefahr:\, U_B>U_R$
    \end{minipage}

    \begin{minipage}{0.2\columnwidth}
        geschlossener Schalter:\\
        $U_B=U_{RL}$\\
        $U_{Schalter}=0$\\
        $I_{RL}\, zu\, beachten$
    \end{minipage}
    \begin{minipage}{0.5\columnwidth}
        \svg[1.0]{diode_schalter_zu}{Diode Schalter zu}{diode_schalter_zu}
    \end{minipage}
    \begin{minipage}{0.3\columnwidth}
        $U_{RL}=U_B-U_S$\\
        $U_{Diode}=U_S$\\
        $Gefahr:\, I_{RL}>I_F$\\
        Überlastung Diode
    \end{minipage}

    $P_{S\max}=Schaltleistung\, der\, Diode=U_{RL}\cdot I_{F\max}\\
    U_{RL}=U_B-U_s$\hspace*{1cm}$I_{F\max}=\frac{P_{tot}}{U_S}\longrightarrow P_{S\max}=P_{tot}(\frac{U_B}{U_S}-1)$
    \hspace*{1cm}$R_{L\min}=\frac{(U_B-U_S)^2}{P_{S\max}}$

\section{Kapazitätsdioden}
    \begin{minipage}{0.3\columnwidth}
        \svg[0.8]{kapazitaetsdiode}{Kapazitätsdiode}{kapazitaetsdiode}
    \end{minipage}
    \begin{minipage}{0.6\columnwidth}
        Verlustfaktor $\tan\delta=\frac{r_s}{X_C}=2\pi f\cdot C_D\cdot r_s$\\
        Güte $Q=\frac{1}{\tan\delta}=\frac{X_C}{r_s}=\frac{1}{2\pi f\cdot C_D\cdot r_s}$\\
        $\quad\quad X_e=(2\pi f\cdot C_D)^{-1}$\\
    \end{minipage}

\section{Bipolartransistoren}
    \svg[0.9]{bjt-pinout}{BJT-Pinout}{bjt-pinout}

\begin{minipage}{0.5\columnwidth}
    Strom und Spannung am NPN-Transistor
\end{minipage}
\begin{minipage}{0.5\columnwidth}
    Strom und Spannung am PNP-Transistor    
\end{minipage}
\noindent\\
$I_B\uparrow\rightarrow I_C\uparrow\quad I_B=0\rightarrow I_C=0$
\renewcommand{\arraystretch}{1.0}
\begin{table}[H]
    \begin{tabularx}{\columnwidth}{l l l}
        Wert                                           & Formel                                                                   & Bemerkung \\
        Gleichstromverstärkung $B$                     & $=\frac{I_C}{I_B}$                                                       & bei $U_{CE}=konst.$ \\
        dyn. Gleichstromverstärkung $\beta$            & $=h_{21e}=\frac{\Delta I_C}{\Delta I_B}$                                 & \\
        $P_V$                                          & $=U_{CE}\cdot I_C + \underbrace{U_{BE}\cdot I_B}_\text{vernachl. klein}$ & \\
        Eingangswiderstand Transistor $r_{BE}$         & $=h_{11e}=\frac{\Delta U_{BE}}{\Delta I_{BE}}$                           & \\
        Ausgangsleitwert Transistor $\frac{1}{r_{CE}}$ & $=h_{22e}=\frac{\Delta I_C}{\Delta U_{CE}=\frac{1}{r_{CE}}}$             & \\
        Steilheit des Transistors $S$                  & $=y_{21}=\frac{\Delta U_{BE}}{\Delta I_{BE}}$                            & \\
        $C_1$                                          & $=\frac{1}{2\pi f_{gu}\cdot (r_e+R_i)}$                                  & \\
        $C_2$                                          & $=\frac{1}{2\pi f_{gu}\cdot (r_a+R_L)}$                                  & \\
        $r_e$                                          & $=R_1\parallel R_2\parallel r_{BE}$                                      & \\
        Übersteuerungsgrad $"u$                        & $=\frac{I_{BX}}{I_{B"U}}$                                                & typisch: $2\dots10$\\
    \end{tabularx}
\end{table}
$I_{BX}=$ Basisstrom bei Übersteuerung, $I_{B"U}=$ Basisstrom an der Übersteuerungsgrenze ($I_{B"U}$ entspricht $I_B$, siehe außerdem \literef{Schaltverstaerker-Hilfsspannungen}), Übersteuerungsgrad heute für Energieeffizienz meist $"u\approx 1.5\dots 2.5$
\subsection{Bestimmung \texorpdfstring{$R_C$}{RC}}
$U_{CE}\approx \frac{1}{2}U_B\quad R_C=\frac{U_B-U_{CE(a)}}{I_{C(a)}}$

\subsection{Erzeugen der Basisvorspannung}
    \begin{minipage}{0.5\columnwidth}
        durch Basisvorwiderstand\\
        $R_B=\frac{U_B-U_{BE(a)}}{I_{B(a)}}$\\ %vorher U_B-U_{CE}, laut Martin falsch
        $R_C=\frac{U_B-U_{CE}}{I_C}$
    \end{minipage}
    \begin{minipage}{0.5\columnwidth}
        \svg[0.8]{basisvorspannung_widerstand}{Basisvorspannung Widerstand}{bv-spannung-widerstand}
    \end{minipage}

    \begin{minipage}{0.5\columnwidth}
        durch Basisspannungsteiler\\
        $I_q=2\dots 10\cdot I_B\\
        R_1=\frac{U_B-U_{BE(a)}}{I_q+I_B}\\
        R_2=\frac{U_{BE(a)}}{I_q}$\\
        $R_C=\frac{U_B-U_{CE}}{I_C}$
    \end{minipage}
    \begin{minipage}{0.5\columnwidth}
        \svg[0.8]{basisvorspannung_spannungsteiler}{Basisvorspannung Spannungsteiler}{bv-spannung-teiler}
    \end{minipage}

\subsection{Thermische Arbeitspunktstabilisierung}
    \begin{minipage}{0.5\columnwidth}
        durch Emitter-Widerstand:\\
        $I_E=I_C+I_B\quad U_{RE}=R_E\cdot I_E\\
        R_C=\frac{U_B-U_{CE}-U_{RE}}{I_C}$\\ %U_BE zu U_RE korrigiert laut Eric + Martin
        $C_E\geq\frac{h_{21e}}{2\pi f_{gu}\cdot(h_{11e}+R_i)}=\frac{\beta}{2\pi f_{gu}\cdot(r_{BE}+R_i)}$ %in FS "h_{11}e", ich nehme an Fehler?
        \svg[0.8]{emitter-widerstand}{Emitter-Widerstand}{emitter-widerstand}
    \end{minipage}
    \begin{minipage}{0.5\columnwidth}
        durch Spannungsgegenkopplung:\\
        $R_1=\frac{U_{CE}-U_{BE(a)}}{I_q+I_B}\quad R_2=\frac{U_{BE}}{I_q}\\
        R_C=\frac{U_B-U_{CE}}{I_C+I_B+I_q}$ 
        \svg[0.8]{spannungsgegenkopplung}{Spannungsgegenkopplung}{spannungsgegenkopplung}
    \end{minipage}
    
    durch Thermistor (NTC=Heißleiter, PTC=Kaltleiter, Bsp. für NTC):\\
    $R_2\parallel\text{Thermistor}\rightarrow R_{2ges\downarrow}=R_2\parallel R_{NTC\downarrow}\rightarrow U_{BE_a\downarrow}$
    \svg[0.3]{thermistor-schaltung}{Thermistor-Schaltung}{thermistorschaltung}
\section{Transistor-Grundschaltungen}
    \subsection{Emitter-Schaltung}
    \renewcommand{\arraystretch}{0.9}
    \begin{table}[H]
        \begin{tabularx}{\columnwidth}{l l l}
            Wert  & Formel                                                  & Bemerkung \\
            $r_e$ & $=R_1\parallel R_2\parallel r_{BE}$                     & wenn $R_E$ mit $C_E$ überbrückt \\
            $r_e$ & $=R_1\parallel R_2\parallel (r_{BE}+\beta\cdot R_E)$    & ohne $C_E$ \\
            $r_a$ & $=R_C\parallel r_{CE}$                                  & \\
            $V_u$ & $=\frac{\beta}{r_{BE}}\cdot r_a$                        & ohne Last \\
            $V_u$ & $=\frac{\beta}{r_{BE}}\cdot (r_a\parallel R_L)$         & mit Last \\
            $V_i$ & $=\beta\cdot\frac{r_{CE}}{r_{CE}+R_C}$                  & ohne Last \\
            $V_i$ & $=\beta\cdot\frac{r_{CE}}{r_{CE}+(R_C\parallel R_L)}$   & mit Last \\
            $V_p$ & $=V_u\cdot V_i\quad \varphi=180^{\circ}$                & \\ %grad per ° ging nicht
            $C_E$ & $=\frac{h_{21E}}{2\pi f_{gu}\dot (h_{11E}+R_i)}$        & \\
            $I_q$ &                                                         & $10\dots100\mu A$\\
        \end{tabularx}
    \end{table}
    \svg[0.35]{emitter-schaltung}{Emitterschaltung}{emitterschaltung}
\subsection{Kollektor-Schaltung}
    Anmerkung: BST = Basisspannungsteiler
    \renewcommand{\arraystretch}{1.1}
    \begin{table}[H]
        \begin{tabularx}{\columnwidth}{l l l}
            Wert         & Formel                                                                       & Bemerkung \\
            $r_{e_{oL}}$ & $=R_1\parallel [r_{BE}+\beta(R_E\parallel r_{CE})]$                          & ohne BST \\ %das hier ohne? in Original beides mit
            $r_{e_{oL}}$ & $=R_1\parallel R_2\parallel [r_{BE}+\beta(R_E\parallel r_{CE})]$             & mit BST \\
            $r_{e_{mL}}$ & $=R_1\parallel [r_{BE}+\beta(R_E\parallel r_{CE}\parallel R_L)]$             & ohne BST \\ %das hier ohne? in Original beides mit
            $r_{e_{mL}}$ & $=R_1\parallel R_2\parallel [r_{BE}+\beta(R_E\parallel r_{CE}\parallel R_L)]$& mit BST \\
            $r_a$        & $=R_E\parallel\frac{r_{BE}*R_i}{\beta}$                                      & \\
            $V_u$        & $=\frac{\beta\cdot R_E}{\beta\cdot R_E+r_{BE}}$                              & ohne Last \\
            $V_u$        & $=\frac{\beta\cdot(R_E\parallel R_L)}{\beta\cdot (R_E\parallel R_L)+r_{BE}}$ & mit Last \\
            $V_i$        & $=\beta\cdot\frac{r_{CE}(1+\beta)}{R_E+r_{CE}}$                              & ohne Last \\
            $V_i$        & $=\beta\cdot\frac{r_{CE}(1+\beta)}{(R_E\parallel R_L)+r_{CE}}$               & mit Last \\
            $V_p$        & $=V_u\cdot V_i\quad \varphi=0^{\circ}$                                       & \\ %grad per ° ging nicht
        \end{tabularx}
    \end{table}
    \svg[0.4]{kollektor-schaltung-martin}{Martins Kollektorschaltung als SVG}{kollektorschaltung}
\subsection{Basis-Schaltung}
    \begin{minipage}{0.3\columnwidth}
        \begin{table}[H]
            \begin{tabularx}{\columnwidth}{l l}
                Wert  & Formel \\
                $r_e$ & $=R_E\parallel\frac{r_{BE}}{\beta}$ \\
                $r_a$ & $=R_C\parallel r_{CE} $ \\
                $V_u$ & $=\frac{\beta}{r_{BE}}\cdot r_a$ \\
                $V_i$ & $=\frac{\beta}{\beta +1}$ \\
                $V_p$ & $=V_u\cdot V_i$\\
                $\varphi$ & $=0^{\circ}$ \\ %grad per ° ging nicht
            \end{tabularx}
        \end{table}
    \end{minipage}
    \begin{minipage}{0.7\columnwidth}
        \begin{circuitikz}
\draw
(0,0) node[](Ue){}
to [open,v=$U_e$] ++(0,-3)
++(0,0) node [](gndleft){}
(Ue) to [capacitor, o-*, C=$C_1$] ++(2,0)
++(0,0) node[](tmp1){}
to [R, *-*, R=$R_E$] ++(gndleft-|gndleft)
to [short, *-o] ++(-2,0)
(tmp1) to [short] ++(4,0) %change this
++(0,0) node[npn, rotate=90, yscale=-1, anchor=emitter](trans){}
(trans.base) to [short, -*] ++(-2,0)
++(0,0) node[](tmp2){}
to [R, *-*, R=$R_2$] (gndleft-|tmp2)
(tmp2) to [short, *-] ++(0,1)
to [R, R=$R_1$] ++(0, 2)
++(0,0) node[](tmp3){}
to [short, -*] (tmp3-|trans.collector)
++(0,0) node[](tmp4){}
to [R, -*, R=$R_C$] (trans.collector)
to [C, *-o, C=$C_3$] ++(2,0)
++(0,0) node[](Ua){}
to [open, v=$U_a$] (gndleft-|Ua)
to [short, o-*] (gndleft-|trans.base)
(tmp4) to [short, *-o] ++(2,0)
++(0,0) node[anchor=west](UB){$U_{B+}$}
(trans.base) to [C, *-*, C=$C_2$] (gndleft-|trans.base)
++(0,0) node[ground](gnd){}
(gndleft-|trans.base) to [short] (gndleft-|tmp2)
to [short] (gndleft-|tmp1)
;
\end{circuitikz}
    \end{minipage}
\subsection{Selektivverstärker} %TODO Schaltung? - keine ahnung. :c ~w
    \begin{minipage}{0.45\columnwidth}
        \renewcommand{\arraystretch}{1.05}
        \begin{table}[H]
            \begin{tabularx}{\columnwidth}{l l}
                Wert            & Formel \\
                $f_o$           & $=\frac{1}{2\pi\sqrt{L\cdot C}}$ \\
                $Q=G\ddot{u}te$ & $=\frac{R\parallel r_{CE}}{X_C}=\frac{R\parallel r_{CE}}{X_L}$ \\
                $X_C$           & $=\frac{1}{2\pi f\cdot C}$ \\
                $X_L$           & $=2\pi f\cdot L$ \\
            \end{tabularx}
        \end{table}
    \end{minipage}
    \begin{minipage}{0.45\columnwidth}
        \renewcommand{\arraystretch}{1.05}
        \begin{table}[H]
            \begin{tabularx}{\columnwidth}{l l}
                Wert        & Formel \\
                $b_{oL}$    & $=\frac{f_0}{Q}=\frac{f_0\cdot X_C}{R\parallel r_{CE}}=\frac{f_0\cdot X_L}{R\parallel r_{CE}}$ \\
                $b_{mL}$    & $=\frac{f_0\cdot X_C}{R\parallel r_{CE}\cdot R_L}=\frac{r_0\cdot X_L}{R\parallel r_{CE}\parallel R_L}$ \\
                $V_u$       & $=Y_{21}(R\parallel r_{CE})$ \\
            \end{tabularx}
        \end{table}
    \end{minipage}
\subsection{Schaltverstärker}
    Schaltfrequenz $f_{\max}=\frac{1}{t_{ein}+t_{aus}}\quad R_1=\frac{U_B-U_{CX}}{I_{BX}}\quad R_2=\frac{U_e-U_{BE}}{I_{BX}}$
    \subsubsection{Schaltverstärker mit Hilfsspannungen}
    \label{Schaltverstaerker-Hilfsspannungen}
        \large
        $I_{BX}=\frac{\ddot{u}\cdot U_B}{R_c\cdot B}\quad U_{IH}\geq R_1(I_{BX}+\frac{U_{BEX}+|U_H|}{R_2})+U_{BEX}\\\\
        \frac{R_1\cdot (U_{BEX}+|U_H|)}{U_{IH}-U_{BEX}-I_{BX}\cdot R_1}\leq R_2\leq\frac{R_1(U_{BEY}+|U_H|)}{U_{IL}-U_{BEY}}$\\\\
        $R_1=\frac{(U_{IH}-U_{BEX})(U_{BEY}+U_H)-(U_{BEX}+U_H)(U_{IL}-U_{BEY})}{I_{BX}\cdot (U_{BEY}+U_H)}$\\\\
        \normalsize
        standardmäßig: $U_{BEX}=0,8V\quad U_{BEY}=0,2V$
\subsection{Feldeffekt-Transistoren}
    Vorwärtssteilheit $y_{21}=\frac{\Delta I_D}{\Delta U_{GS}}=S$\quad Ausgangsleitwert $y_{22}=\frac{\Delta I_D}{\Delta U_{DS}}=\frac{1}{r_{DS}}$\\
    Automatische Gate-Spannungs-Erzeugung $U_{RS}=I_D\cdot R_S\quad P_{tot}=U_{DS}\cdot I_D$\\
    Gatespannungserzeuger mit Gatespannungsteiler $R_1=\frac{U_B-U_{GS}-U_{RS}}{I_q}\quad R_2=\frac{U_{GS}+U_{RS}}{I_q}$
    \svg[0.7]{fets}{FETs}{fets}
    \subsubsection{Source-Schaltung}
        \begin{minipage}{0.5\columnwidth}
            \renewcommand{\arraystretch}{1.01}
            \begin{table}[H]
                \begin{tabularx}{\columnwidth}{l l l}
                    Wert     & Formel                                                           & Bermerkung\\
                    $C_S$    & $=0,2\cdot\frac{y_{21}}{f_{gu}}$                                 & \\ %!korrigiert von y22, Hinweis von Martin
                    $C_1$    & $=\frac{1}{2\pi f_{gu}(R_i+r_e)}$                                & \\
                    $C_2$    & $=\frac{1}{2\pi f_{gu}(R_L+r_a)}$                                & \\
                    $R_G$    & $=\frac{U_{GS}=0,5V}{I_{GSS}}$                                   & größere $I_{GSS}$ wählen \\
                    $R_{GS}$ & $=\frac{U_{GS}}{I_{GSS}}$                                        & gr. $U_{GS}$, kl. $I_{GSS}$ wählen \\
                    $R_S$    & $=\frac{U_{GS}}{I_D}=\frac{U_{RS}}{I_D}$                         & kleine $U_{GS}$ wählen \\
                    $r_a$    & $=R_D\parallel r_{DS}$                                           & \\
                    $r_{DS}$ & $=\frac{1}{y_{22}}$                                              & \\
                    $R_D$    & $=\frac{U_B-U_{DS}-[U_{RS}\, o.\, U_{GS}]}{I_D}=\frac{U_D}{I_D}$ & \\
                    $r_e$    & $=\frac{R_G\cdot R_{GS}}{R_G+R_{GS}}=R_G\parallel R_{GS}$        & oder:\\
                    $r_e$    & $=R_1\parallel R_2\parallel R_{GS}$                              & für Gatesp. Teiler \\
                    $V_U$    & $=y_{21}\cdot r_a$                                               & \\
                    $U_{RG}$ & $=-I_{GSS}\cdot R_G\quad\varphi=180^{\circ}$                     & \\ %grad per ° ging nicht
                \end{tabularx}
            \end{table}
        \end{minipage}
        \begin{minipage}{0.5\columnwidth}
            \vspace*{0.5cm}
            \svg[1.0]{source-schaltung}{Source-Schaltung}{sourceschaltung}
        \end{minipage}
    \subsubsection{Drain-Schaltung}
        $r_e=R_{GS}(1+y_{21}\cdot R_S)\parallel R_G$ bzw. bei Gatespannungsteiler: $r_e=R_{GS}(1+y_{21}\cdot R_S)\parallel R_1\parallel R_2$ \\
        $r_a=R_S\parallel \frac{1}{y_{21}}\quad V_U=\frac{y_{21}\cdot R_S}{1+y_{21}\cdot R_S}\: \varphi=0^{\circ}$ \\ %grad per ° ging nicht
        %TODO @colin "folgende Zeile weird, da sind wahrscheinlich Fehler" - ignorieren weil mir auch zu hoch ~c
        $R_G=\frac{-U_{GSS}}{I_{GSS}}\quad
        R_S=\frac{U_{RS}}{I_D}\quad
        R_{GS}=\frac{-U_{GS}}{-I_{GSS}}\quad
        R_1=\frac{U_B-U_{GS}-U_S}{I_q}\quad
        R_2=\frac{U_{GS}+U_S}{I_q}$ %R1, R2 laut Kevin
        \svg[1.0]{drain-schaltung}{Drainschaltung}{drainschaltung}
\section{OPV - Operationsverstärker}
    \begin{minipage}{0.5\columnwidth}
    \renewcommand{\arraystretch}{1.05}
    \begin{table}[H]
        \begin{tabularx}{\columnwidth}{l l}
            Wert                & Formel \\
            $V_{ldb}$           & $=20\lg\frac{U_a}{U_e}$ (Verstärkung)\\
            $I_K$               & $=\frac{U_2-U_1}{Z_K}\quad I_1+I_K=0$\\
            $\underline{Z_1}$   & $=\frac{\underline{U_1}}{\underline{I_1}}=\frac{\underline{U_1}}{\underline{-I_K}}$\\
            $\underline{Z_1}$   & $=\frac{Z_K}{-V_0+1}$\\
            $U_1$               & $=\frac{U_2}{V_0}$ \\
            $z_2$               & $=\frac{z_K}{1+\frac{1}{-V_0}}$
        \end{tabularx}
    \end{table}
\end{minipage}
\begin{minipage}{0.5\columnwidth}
    \bild[1.0]{opv.png}{Platzh. für gut}{opv}
\end{minipage}
\subsection{Anwendungen mit frequenzunabhängiger Gegenkopplung}
    \subsubsection{Invertierender Verstärker/Umkehrverstärker}
        \begin{minipage}{0.6\columnwidth}
            \renewcommand{\arraystretch}{1.05}
            \begin{table}[H]
                \begin{tabularx}{\columnwidth}{l l}
                    Wert    & Formel \\
                    $r_e$   & $=R_1$ \\
                    $r_a'$  & $=r_a\cdot\frac{V}{V_0}$\\
                    $U_a$   & $=-\frac{R_2}{R_1}\cdot U_e$ \\
                    $V$     & $=-\frac{R_2}{R_1}=\frac{U_a}{U_e}$ \\
                \end{tabularx}
            \end{table}
            Einfluss der endlichen Verstärkung\\auf einen realen OPV:\\
            $V=-\frac{V_0\cdot R_2}{R_2+R_1\cdot V_0+R_1}$
        \end{minipage}
        \begin{minipage}{0.4\columnwidth}
            \bild[1.0]{opv-umkehrverst.png}{Platzh. für gut}{umkehrverstärker}
        \end{minipage}
    \subsubsection{Nichtinvertierter Verstärker/Elektrometerverstärker}
        \begin{minipage}{0.6\columnwidth}
            \renewcommand{\arraystretch}{1.1}
            \begin{table}[H]
                \begin{tabularx}{\columnwidth}{l l}
                    Wert  & Formel \\
                    $V$   & $=1+\frac{R_2}{R_1}=\frac{U_a}{U_e}$ \\
                    $U_a$ & $=(1+\frac{R_2}{R_1})\cdot U_e$ \\
                    $R_2$ & $=\longrightarrow R_K$\\
                    $r_e$ & $=r_{GL}=r_{aL}\approx 0\Omega$ \\
                    $r_a'$& $=r_a\cdot\frac{v}{v_0}$\\
                \end{tabularx}
            \end{table}
        \end{minipage}
        \begin{minipage}{0.4\columnwidth}
            \bild[1.0]{opv-elmverst.png}{Platzh. für gut}{elmverst}
        \end{minipage}
    \subsubsection{Spannungsfolger}
        \begin{minipage}{0.6\columnwidth}
            \begin{table}[H]
                \begin{tabularx}{\columnwidth}{l l}
                    Wert  & Formel \\
                    $U_a$ & $=U_e$ \\
                    $V$   & $=1$ \\
                \end{tabularx}
            \end{table}
        \end{minipage}
        \begin{minipage}{0.4\columnwidth}
            \bild[1.0]{opv-spannungsfolger.png}{Platzh. für gut}{spannungsfolger}
        \end{minipage}
    \subsubsection{(invertierender) Addierverstärker}
        \begin{minipage}{0.6\columnwidth}
            Alle Addierverstärker sind invertierend. $\longrightarrow\ -U_a$
            \renewcommand{\arraystretch}{1.1}
            \begin{table}[H]
                \begin{tabularx}{\columnwidth}{l l}
                    Wert            & Formel \\
                    \vspace{-2ex} % zweite Formel zu -Ua trotz overbrace nach oben ziehen
                    $-U_a$          & $=(I_1+I_2)\cdot R_0$\\
                                    & $=\frac{R_0}{R_1}\cdot U_{e1}+\frac{R_0}{R_2}\cdot U_{e2}+\overbrace{\frac{R_0}{R_\text{n}}\cdot U_{e\text{n}}}^{\text{nur für weitere }U_{e\text{n}}}$ \\
                    $-U_a$          & $=(U_{e1}+U_{e2})\cdot\frac{R_0}{R_1}\leftarrow$ nur für $R_1=R_2$ !\\
                    $-I_K$          & $=I_1+I_2$\\
                    $V$             & $=1$ \\
                    $U_{e\text{n}}$ & $=I_\text{n}\cdot R_\text{n}$ \\
                \end{tabularx}
            \end{table}
        \end{minipage}
        \begin{minipage}{0.4\columnwidth}
            \bild[1.0]{opv-addverst.png}{Platzh. für gut}{addierverstärker}
        \end{minipage}
    \subsubsection{(invertierender) Subtrahierverstärker}
        \begin{minipage}{0.6\columnwidth}
            \begin{table}[H]
                \begin{tabularx}{\columnwidth}{l l}
                    $-U_a$   & $=\frac{R_K}{R_1}\cdot U_{e1}-(1+\frac{R_K}{R_1})\cdot U_{e2}\underbrace{\cdot \frac{R_3}{R_2+R_3}}_\text{falls $R_3$ vorhanden}$ \\
                    $U_a$    & $=U_{a1} + U_{a2}$ \\
                    $-U_{a1}$& $=\frac{R_k}{R_1}\cdot U_{e1}$ \\
                    $U_{a2}$ & $=(1+\frac{R_k}{R_1})\cdot U_{e2}\underbrace{\cdot\frac{R_3}{R_2+R_3}}_\text{falls $R_3$ vorhanden}$ \\
                \end{tabularx}
            \end{table}
        \end{minipage}
        \begin{minipage}{0.4\columnwidth}
            \bild[1.0]{opv-subverst.png}{Platzh. für gut}{subtrahierverstärker}
        \end{minipage}
\subsection{OPV mit frequenzunabhängiger Gegenkopplung}
    \begin{minipage}{0.5\columnwidth}
        \subsubsection{Integrierverstärker}
        $\frac{1}{R\cdot C}\cdot U_e = - \frac{d\, u_a}{dt}$\\
        $-U_a=\frac{1}{R\cdot C}\cdot \int U_e(t) dt+U_0$\\ %korrektur laut OPV Stand 060321
        \bild[0.8]{opv-intverst.png}{Platzh. für gut}{integrierverstärker}
    \end{minipage}
    \begin{minipage}{0.5\columnwidth}
        \subsubsection{Differenzierer}
        $i_c=C\cdot\frac{du_c}{dt}\quad i_k=\frac{U_a}{R}=-i_c$\\
        $-U_a=R\cdot C\cdot\frac{d u_e}{dt}=R\cdot C\cdot\frac{\Delta u_e}{\Delta t}$\\
        \bild[0.8]{opv-differenzierer.png}{Platzh. für gut}{differenzierer}
    \end{minipage}
    \subsubsection{Komparator/Vergleicher}
        \begin{center}
            invertierter Komparator / nicht-invertierter Komparator
            \bild[1.0]{opv-komparator.png}{Platzh. für gut}{komparator}
        \end{center}
\subsection{Schmidtt-Trigger (Schwellwertschalter)}
    \begin{minipage}[t][0.4\pdfpageheight][t]{0.5\columnwidth} %extra Argumente erzeugen 0.4xPapierhöhe hohe, top-aligned..te minipage
        \subsubsection{invertierter SWS}
        \renewcommand{\arraystretch}{1.1}
        \begin{table}[H]
            \begin{tabularx}{\columnwidth}{l l}
                Wert      & Formel \\
                $U_{ein}$ & $=\frac{R_2}{R_1+R_2}\cdot U_{a-}$ \\
                $U_{aus}$ & $=\frac{R_2}{R_1+R_2}\cdot U_{a+}$ \\
                $U_{Hys}$ & $=U_{aus}-U_{ein}$
            \end{tabularx}
        \end{table}
        \bild[0.8]{schmidtt-inv.png}{Platzh. für gut}{sws-inv}
    \end{minipage}
    \begin{minipage}[t][0.4\pdfpageheight][t]{0.5\columnwidth}
        \subsubsection{nichtinvertierter SWS}
        \renewcommand{\arraystretch}{1.1}
        \begin{table}[H]
            \begin{tabularx}{\columnwidth}{l l}
                Wert      & Formel \\
                $U_{ein}$ & $=-\frac{R_1}{R_2}\cdot U_{a-}$ \\
                $U_{aus}$ & $=-\frac{R_1}{R_2}\cdot U_{a+}$ \\
            \end{tabularx}
        \end{table}
        \bild[0.8]{schmidtt-ninv.png}{Platzh. für gut}{sws-ninv}
    \end{minipage}
    \begin{minipage}[t][0.4\pdfpageheight][t]{0.5\columnwidth}
        \subsubsection{nichtinvertierter SWS mit Referenzspannungsquelle}
        \renewcommand{\arraystretch}{1.1}
        \begin{table}[H]
            \begin{tabularx}{\columnwidth}{l l}
                Wert      & Formel \\
                $U_{ein}$ & $=-\frac{R_1}{R_2}\cdot U_{a\min}+U_{ref}(1+\frac{R_1}{R_2})$\\
                $U_{aus}$ & $=-\frac{R_1}{R_2}\cdot U_{a\max}+U_{ref}(1+\frac{R_1}{R_2})$\\
            \end{tabularx}
        \end{table}
        \bild[0.8]{schmidtt-ninv-ref.png}{Platzh. für gut}{sws-ninv-ref}
    \end{minipage}
    \begin{minipage}[t][0.4\pdfpageheight][t]{0.5\columnwidth}
        \subsubsection{invertierter SWS mit Referenzspannungsquelle}
        \renewcommand{\arraystretch}{1.1}
        \begin{table}[H]
            \begin{tabularx}{\columnwidth}{l l}
                Wert      & Formel \\ %!beide Formeln in Original mit "-" vorn dran, laut Hannes nicht
                $U_{ein}$ & $=\frac{R_2}{R_1+R_2}\cdot U_{a\min}+U_{ref}(1-\frac{R_2}{R_1+R_2})$\\
                $U_{aus}$ & $=\frac{R_2}{R_1+R_2}\cdot U_{a\max}+U_{ref}(1-\frac{R_2}{R_1+R_2})$\\
            \end{tabularx}
        \end{table}
        \bild[0.8]{schmidtt-inv-ref.png}{Platzh. für gut}{sws-inv-ref}
    \end{minipage}
\end{document}